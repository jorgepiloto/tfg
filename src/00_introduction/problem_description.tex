% --- PROBLEM DESCRIPTION --- %
\section{Problem description and motivation}

The two-body boundary value problem in the framework of orbital mechanics and
astrodynamics is known as the Lambert's problem. Solutions to this problem find
the orbit which connects two known position vectors over a finite time of
flight.

\begin{figure}[H]
  \centering
  \includegraphics[scale=1]{problem_geometry.png}
  \caption[The geometry of Lambert's problem]{
    The geometry of Lambert's problem as seen from an inertial frame
    centered in the orbit's attractor. Observation vectors are labeled as
    $\vec{r_{1}}$ and $\vec{r_{2}}$, being $\Delta \theta$ the angle between
    them and $\Delta t$ the epoch difference.
  }
  \label{fig:problem_geometry}
\end{figure}

Since its formulation almost 300 years ago, a plethora of solutions have been
proposed. Nevertheless, the problem is still of interest due to its applications
(see section \ref{sec:applications}) and many modern authors have addressed it
by developing new numerical techniques. Therefore, when it comes to solving for
Lambert's problem, a big set of algorithms is available and the following
question arises: which one of those performs the best under particular given
conditions. The search for a solution to the previous question is the motivation
behind this work.
