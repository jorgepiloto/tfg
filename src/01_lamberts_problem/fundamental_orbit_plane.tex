\section{The orbital plane}

Assuming that the time of flight is not null, and that the position vectors are
not coincident neither forming an angle of $180$ degrees between them, the whole
orbit movement will be contained within a plane. This plane is usually named the
orbital plane. The red colored plane in figure
\ref{fig:problem_geometry} represents it.

For the case of the Lambert's problem, the orbital plane can be computed
from the initial and final position vectors. These two vectors form a basis from
which any other vector contained in the plane can be obtained from the relation
shown in equation \ref{eq:linear_combination}:

\begin{equation}
  \vec{r} = a \vec{r_{1}} + b \vec{r_{2}}
  \label{eq:linear_combination}
\end{equation}

where $a$ and $b$ are two known real values.

As long as both vectors are linearly independent, the orbital plane is defined.
This implies that the cross product $\vec{r_{1}} \times \vec{r_{2}}$ must be not
null, otherwise singularities arise. This justifies previous assumption about
not having two position vectors which might be parallel (i.e. forming $0$ or
$180$ degrees).

The orientation of the orbital plane in space with respect to (w.r.t.) is given
by the right ascension of the ascending node (RAAN) $\Omega$ and the inclination
$i$. Therefore, it is interesting to analyze the Lambert's problem by making use
of the COE set instead of the RV one. However, once the COE elements are known,
a conversion to the RV ones is required, as the initial elements given in the
statement of the problem are the position ones.

Because both the RAAN and the inclination are computed from a vector in the
direction and sense of the specific angular momentum of the orbit, $\vec{h}$,
the following subsection presents a simple procedure on how to solve for this
particular vector.


\subsection{Solving for the unitary specific angular momentum vector}

It is not possible to compute the specific angular momentum of the orbit because
a velocity at a particular point would be required. This condition would satisfy
the RV set, meaning that the orbit is fully determined. However, a unitary
vector in the direction and sense of the angular momentum can be computed making
use of the two knwon position vectors and the sense of motion of the observed
body.

The sense of motion (prograde or retrograde)\footnote{Prograde and retrograde
  orbits are defined according if their inclinations are lower or greater than
  $90$ degrees respectively.} is an important parameter, since it fixes the sense
of the angular momentum of the orbit. The reason is that, when computing a
normal vector to the normal plane via equation \ref{eq:normal_vector}

\begin{equation}
  \vec{h_{0}} = \frac{\vec{r_{1}} \times \vec{r_{2}}}{\norm{\vec{r_{1}}}\norm{\vec{r_{2}}}}
  \label{eq:normal_vector}
\end{equation}

the resultant normal vector has always the sense imposed by the so-called
\textit{right hand rule} according to the shortest angle between $\vec{r_{1}}$
and $\vec{r_{2}}$. In addition, notice that vector $\vec{h_{0}}$ goes in the
direction of the $\vec{h}$ vector, but does not have neither its sense or
magnitude.

Therefore, the value of $\vec{h_{0}}$ needs to be corrected according if it does
not agree with a prograde/retrograde type of motion observed. This implies
checking for the sign of the third component of via $h_{0_{Z}} =
  \vec{k} \cdot \vec{h_{0}}$ and apply:

\vspace{0.15cm}
\bgroup
\def\arraystretch{1.5}
\begin{table}[H]
  \centering
  \begin{tabular}{|c|c|c|}
    \hline
    \textbf{Sign of the vertical component} & \textbf{Prograde} & \textbf{Retrograde} \\ \hline
    $h_{0_{Z}} > 0$                         & $\vec{h_{0}}$     & $-\vec{h_{0}}$      \\ \hline
    $h_{0_{Z}} < 0$                         & $-\vec{h_{0}}$    & $\vec{h_{0}}$       \\ \hline
  \end{tabular}
  \caption{Sign correction for the normal specific angular momentum vector.}
\end{table}
\egroup

where vector $\vec{k}$ in the dot product operation is a unitary one in the
direction and sense of the positive $Z$ axis of the inertial reference frame
selected to study the problem.

With previous correction, the resultant vector $\vec{h_{0}}$ goes in the
direction and sense of the specific angular momentum one but has unitary
modulus.

\subsection{Solving for the inclination of the orbit}

Once $\vec{h_{0}}$ has been computed and its sign properly fixed, the
inclination of the orbit can be computed by making use of expression
\ref{eq:inclination_orbit}:

\begin{equation}
  i = \arccos{\left(\frac{h_{0_{Z}}}{\norm{\vec{h_{0}}}} \right)}
  \label{eq:inclination_orbit}
\end{equation}

Because the sign correction was previously applied, the value of the inclination
must agree with the prograde/retrograde type of motion. Furthermore, the output
of the $\arccos$ function is bounded to the inclination domain (i.e. $0$ and
$\pi$ radians).


\subsection{Solving for the RAAN of the orbit}

Finally, the right ascension of the ascending node is computed as usual. First,
a vector in the direction of the line of nodes is computed taking advantage of
the expression \ref{eq:line_of_nodes}:

\begin{equation}
  \vec{n} = \vec{k} \times \vec{h_{0}}
  \label{eq:line_of_nodes}
\end{equation}

With previous vector, it is possible now to compute the RAAN by applying
equation \ref{eq:raan}

\begin{equation}
  \Omega = \arccos{\left(\frac{\vec{i} \cdot \vec{n}}{\norm{\vec{i}}
      \norm{\vec{n}}}\right)}
  \label{eq:raan}
\end{equation}

where vector $\vec{i}$ in the dot product operation is a unitary one in the
direction and sense of the positive $X$ axis of the inertial reference frame
selected to study the problem.

Because the RAAN is bounded between $[0,2\pi)$, a correction might be required,
as the $\arccos$ function only returns values between $[0,\pi)$. This correction
is triggered if the lateral component of the line of nodes, $\vec{n_{j}} < 0$,
so the final value of $\Omega$ becomes:

\begin{equation}
  \Omega = 2\pi - \Omega \Rightarrow \vec{n_{j}} < 0
  \label{eq:raan_correction}
\end{equation}

From the point of view of a computer routine, it is better to make use of the
$\arctantwo$\footnote{This routine was originally introduced by the FORTRAN
  language under the name of $\atantwo$.}, which returns the proper quadrant
between the input coordinates, so the RAAN can be easily computed following
relation \ref{eq:raan_computer}:

\begin{equation}
  \Omega = \arctantwo(n_{y}, n_{x})
  \label{eq:raan_computer}
\end{equation}

where $n_{y}$ and $n_{x}$ are the lateral and longitudinal coordinates of the
line of nodes vector.

Once $\Omega$ has been solved together with $i$, the complete orientation in
space of the orbital plane is known.

\subsection{Solving for the transfer angle}

The computation of the transfer angle requires the knowledge of the sense of
motion (prograde or retrograde) of the orbiting body. By taking advantage of the
definition of the dot product, it is possible to solve the angle between the
initial and final position vector such us:

\begin{equation}
  \Delta \theta_0 = \arccos{\left(\frac{\vec{r_1} \cdot \vec{r_2}}{\norm{\vec{r_1}} \norm{\vec{r_2}}}\right)}
  \label{eq:transfer_angle}
\end{equation}

However, notice that equation \ref{eq:transfer_angle} will always output an
in the range of $[0, \pi]$ radians. A correction, similar to the one applied to
the normal vector needs to be performed according to the sense of motion:


\begin{equation}
  \Delta \theta =
  \begin{cases}
    \Delta \theta_0         & \text{if prograde and $h_{0_{z}} > 0$}   \\
    2 \pi - \Delta \theta_0 & \text{if prograde and $h_{0_{z}} < 0$}   \\
    2 \pi - \Delta \theta_0 & \text{if retrograde and $h_{0_{z}} > 0$} \\
    \Delta \theta_0         & \text{if retrograde and $h_{0_{z}} < 0$} \\
  \end{cases}
\end{equation}


A graphical representation of the geometry of the transfer angle is discussed by
\cite{der2011}, who devotes a whole section for this important but poorly
covered topic in Lambert's literature.

\subsection{The problem as seen from the orbital plane}

Because the orbital plane is completely known, only the following parameters are
still required to solve for the orbit: $a$, $e$, $w$ and $\nu$. Notice that the
true anomaly, that is $\nu$, only gives information about the current location
of the body along its orbit but not about the shape of the orbit itself.
Therefore, the problem is just devoted to find the values of $a$, $e$ and $w$.

Since all the motion of the problem is contained in the orbital plane, it is
feasible to study the Lambert's problem a reference who's $z=0$ plane is
coincident with previous plane. Reader might have thought about using the
\textit{perifocal coordinate system} (PQW) but the argument of perigee is still
an unknown of the problem, so it is not possible to use the PQW frame. With
previous assumption, we can simplify the geometry given by figure
\ref{fig:problem_geometry} into the one shown by figure
\ref{fig:lambert_problem_pqw}.

\vspace{0.15cm}
\begin{figure}[H]
  \centering
  \includegraphics[scale=1]{lambert_problem_pqw.png}
  \caption{
    The geometry of the Lambert's problem as seen from an inertial frame who's
    $XY$ plane coincides with the orbital one..
  }
  \label{fig:lambert_problem_pqw}
\end{figure}

Notice that the attracting body in figure \ref{fig:lambert_problem_pqw} is
located in the known focus $F$, following the two-body dynamics assumption. The
other focus $F'$ is still unknown and needs to be solved to retrieve the full
geometry of the orbit. This also introduces the unknown of the $w$, as the line
where for the semi-major needs to be guessed.


