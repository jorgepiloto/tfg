\section{Geometry of the solutions}

Before introducing or proposing any mathematical method for solving equation
\ref{eq:visviva_series}, we will assume that we managed somehow to obtain the
right value for the semi-major axis of the orbit.

With this value, it is possible to identify the type of orbit, according to the
following table:

\begin{table}[H]
  \centering
  \begin{tabular}{|c|c|c|}
    \hline
    \textbf{$0 < a < \infty$} & \textbf{$a = \infty$} & \textbf{$a < -\infty$} \\ \hline
    Elliptic                  & Parabolic             & Hyperbolic             \\ \hline
  \end{tabular}
  \caption{Type of orbit according to semi-major axis value.}
\end{table}

Knowing the value of $a$ it is possible to estimate the rest of the orbital
elements so the observed orbit is fully determined. However, each type of orbit
requires from its own solution procedure when determining the eccentricity,
argument of perigee and the current true anomaly.

Therefore, in the following subsections, the possible solutions and procedures
for each type of orbit are exposed from a geometrical point of view.

\subsection{The elliptic transfer geometry}

An elliptic orbit has a semi-major axis who's value lies between $[0,\infty)$
units of length. By recalling that:

\begin{equation}
  \overline{FP} + \overline{F'P} = 2a
  \label{eq:ellipse_property}
\end{equation}

being $\overline{FP}$ and $\overline{F'P}$ the distance to the
primary\footnote{The primary focus is the one in which the attractor is placed.}
and vacant\footnote{Named like this because it does not hold the attractor.}
focus, it is possible to find the location of this last one.

In order to achieve previous task, we make use of equation
\ref{eq:ellipse_property} and evaluate it for each known position vectors:

\begin{equation}
  r'_{1} = 2a - r_{1}\quad\quad
  r'_{2} = 2a - r_{2}
\end{equation}

The values $r'_{1}$ and $r'_{2}$ indicate the distance at which the vacant focus
$F'$ is located. Therefore, by tracing a circumference of radius $R_{1} = r'_{1}$
centered at the tip of vector $\vec{r_{1}}$ and another with $R_{2} = r'_{2}$ at
the tip of vector $\vec{r_{2}}$ it is possible to find graphically where the
vacant focus is located.

\subsubsection{Two vacant focus}
When both circumferences cross at two points, meaning there are two possible
vacant focus, as shown by figure \ref{fig:elliptic_geometry_1}. Having two focus
means having two elliptic paths and a total of four orbits:

\begin{itemize}
  \item Red orbit and prograde motion.
  \item Red orbit and retrograde motion.
  \item Blue orbit and prograde motion.
  \item Blue orbit and retrograde motion.
\end{itemize}

By imposing the sense of motion to be prograde or retrograde, the solution space
is reduced to two possible orbits. To distinguish between the two, we introduce
a new parameter named \textit{the orbit path}, which enables us to select
between one or the other. This parameter was introduced by \cite{sun1977}.

In this scenario, it holds that:

\begin{equation}
  a > \frac{\norm{\vec{r_{1}}} + \norm{\vec{r_{2}}} + \norm{\vec{c}}}{4}
\end{equation}

Finally, it must be said that these solution only appear in the
multi-revolution problem.

\begin{figure}[H]
  \centering
  \includegraphics[scale=1]{geometry_elliptic1.png}
  \caption[Elliptic transfer geometry for two vacant focus]{
    In this scenario, there are a total of four solutions considering two
    possible orbits (blue and red) and the transfer angle (lower or greater than
    180 degrees), which is equivalent to prograde or retrograde motion. Fixing
    the type of motion together with the orbit path parameter will retrieve the
    current solution to the problem.
  }
  \label{fig:elliptic_geometry_1}
\end{figure}

\subsubsection{One vacant focus}
In this case, both circumferences are tangent to each other, meaning there is
only one common point and thus a single vacant focus. See figure
\ref{fig:elliptic_geometry_2} for a graphical representation.

In this case, only two solutions can be achieved:

\begin{itemize}
  \item Red orbit and prograde motion.
  \item Red orbit and retrograde motion.
\end{itemize}

Similarly to previous case, imposing the sense of motion will retrieve the
current solution to the problem. Notice now that the orbit path parameter is no
longer required as only one orbit (red one) exists.


For this scenario, the semi-major axis is found to be:
\begin{equation}
  a = \frac{\norm{\vec{r_{1}}} + \norm{\vec{r_{2}}} + \norm{\vec{c}}}{4}
\end{equation}

A single vacant focus is likely to appear for the direct arc transfer problem.

\begin{figure}[H]
  \centering
  \includegraphics[scale=1]{geometry_elliptic2.png}
  \caption[Elliptic transfer geometry for one vacant focus]{
    In this scenario, there are only two solutions considering a single orbit
    (red one) and the transfer angle (lower or greater than 180 degrees),
    equivalent to prograde or retrograde motion. Fixing the type of motion will retrieve the current solution to the problem.
  }
  \label{fig:elliptic_geometry_2}
\end{figure}


\subsubsection{No vacant focus}
In this last situation, there is not a common point between the two
circumferences, meaning that the transfer orbit does not exist for that
particular geometry.

In fact, the semi-major axis turns out to be:
\begin{equation}
  a < \frac{\norm{\vec{r_{1}}} + \norm{\vec{r_{2}}} + \norm{\vec{c}}}{4}
\end{equation}

meaning that the problem is unsolvable.

\subsubsection{Computing the rest of the elements in the elliptic case}
Once the location of the vacant focus has been identified, it is possible to
solve for the focal distance $c$ by applying equation \ref{eq:focal_distance}:

\begin{equation}
  c = \frac{\overline{FF'}}{2}
  \label{eq:focal_distance}
\end{equation}

from which the semi-minor axis $b$ can be obtained:

\begin{equation}
  b = \sqrt{a^2 - c^2}
  \label{eq:focal_distance}
\end{equation}

Since now both axes, the semi-major and semi-minor are known, the eccentricity
of the orbit can be computed using \ref{eq:ellipse_ecc}:

\begin{equation}
  e = \sqrt{1 - \frac{a}{b}} = \frac{\overline{FF'}}{2a}
  \label{eq:ellipse_ecc}
\end{equation}

Notice all previous equations can be applied with $\overline{FF''}$ instead of
$\overline{FF'}$.

Finally, the last element, that is the argument of perigee, can be computed
knowing the angle between the initial position vector and the line of nodes such
that:

\begin{equation}
  w = \arccos{\left(\frac{\vec{n} \cdot \vec{r_{1}}}{\norm{\vec{n}}
      \norm{\vec{r_{1}}}} \right) + \angle P_{1}FF' \pm \pi}
  \label{eq:ellipse_w}
\end{equation}

where in equation \ref{eq:ellipse_w}, $P_{1}$ is the point given by the tip of
vector $\vec{r_{1}}$.

\subsection{The parabolic transfer geometry}

Any point which belongs to a parabola holds the following relation:

\begin{equation}
  \overline{FP} = \overline{lP}
  \label{eq:parabola_property}
\end{equation}

where in previous equation, $l$ is the \textit{directrix} of the parabola.
Equation \ref{eq:parabola_property} shows that the distance from any point of
the parabola is equal to the distance of that point to the directrix.

Therefore, to locate the directrix of the transfer parabola, drawing two
circumferences of radius $R_{1} = \norm{\vec{r_{1}}}$ and
$R_{2}=\norm{\vec{r_{2}}}$ centered at the tip of vectors $\vec{r_{1}}$ and
$\vec{r_{2}}$ respectively. The directrix is given by the tangent lines between
these circumferences. The resultant graphic procedure is shown by figure
\ref{fig:parabolic_geometry}:

\begin{figure}[H]
  \centering
  \includegraphics[scale=1]{geometry_parabolic.png}
  \caption[Parabolic transfer geometry.]{ For the parabolic scenario two possible solutions are available,
    even if there exist two orbits. This is because parabolic orbits are
    not closed and the path can only be followed in one sense. Therefore,
    fixing retrograde or prograde motion fixes the solution to the red or
    blue orbits.
  }
  \label{fig:parabolic_geometry}
\end{figure}

In this case, the exact value for the semi-major axis of the parabola is known,
being $a = \pm \infty$ together with the eccentricity $e=1.00$. Therefore, the
only left parameter is the argument of perigee, which can be computed

\begin{equation}
  w = \arccos{\left(\frac{\vec{n} \cdot \vec{r_{1}}}{\norm{\vec{n}}
      \norm{\vec{r_{1}}}} \right) + \angle P_{1}FD'}
  \label{eq:parabola_w}
\end{equation}

where in equation \ref{eq:parabola_w} the value $D$ refers to the shortest
distance between the focus of the parabola $F$ and its directrix.

\subsection{The hyperbolic transfer geometry}

Similarly to the elliptic case, for the hyperbola the geometric condition that
any point on the curve fulfills is given by equation
\ref{eq:hyperbola_property}:

\begin{equation}
  \overline{FP} - \overline{F'P} = 2a
  \label{eq:hyperbola_property}
\end{equation}

Therefore, the distance to the vacant focus can be computed as:

\begin{equation}
  r'_{1} = 2a + r_{1}\quad\quad
  r'_{2} = 2a + r_{2}
\end{equation}

Again, previous values $r'_{1}$ and $r'_{2}$ are the radius of the
circumferences which must be centered at the tip of vectors $\vec{r_{1}}$ and
$\vec{r_{2}}$ to properly locate the vacant focus.


\vspace{0.15cm}
\begin{figure}[H]
  \centering
  \includegraphics[scale=1]{geometry_hyperbolic.png}
  \caption[Hyperbolic transfer geometry]{For the hyperbolic transfer two solutions are available for a total
    of two orbits, blue and red ones. This is similar to the parabolic case, because the
    orbit is not closed. Notice that only the paths of the hyperbolas which pass trough the
    position vectors are the ones where the orbiting body could be found.
  }
  \label{fig:hyperbolic_geometry}
\end{figure}

For the hyperbolic orbit, both previous circumferences always meet at two
different points, as seen in figure \ref{fig:hyperbolic_geometry}. This is
because:

\begin{equation}
  r'_{1} + r'_{2} = 4a + \norm{\vec{r_{1}}} + \norm{\vec{r_{2}}} > c
\end{equation}

For solving the rest of the elements, and similarly to the elliptic transfer
orbit, the following relations can be applied:

\begin{equation}
  e = \frac{\overline{FF'}}{2a}
  \label{eq:ellipse_ecc}
\end{equation}

and:

\begin{equation}
  w = \arccos{\left(\frac{\vec{n} \cdot \vec{r_{1}}}{\norm{\vec{n}}
      \norm{\vec{r_{1}}}} \right) + \angle P_{1}FF'}
  \label{eq:hyperbola_w}
\end{equation}


\subsection{Singularities of the problem}

The Lambert's problem presents some singularities for particular. That is the
case for non-independent position vectors, meaning that the transfer angle
between them is either $0$ or $180$ degrees.

In this case, equation \ref{eq:normal_vector} becomes null and the direction of
the normal vector to orbit plane cannot be properly determined. Not only that,
equation \ref{eq:inclination_orbit} also suffers from a singularity, as
$\norm{\vec{h_{0}}}=0$ is in the denominator of the expression. Also, the line
of nodes and $\Omega$ can not be computed with the available data.

Rectilinear orbits also produce singularities, as they move along a straight
path for which all the position vectors are proportional. For these type of
orbits, velocity of the orbiting body goes in the direction of the
attracting force during the whole motion. From the point of view of real-world
cases, this scenario appears when a body is released without initial velocity,
so it falls onto its attractor.

All previous singularities are caused by a common reason: a non linearly
independent input which makes corner cases to appear. To avoid those

\begin{itemize}
  \item For two equal vectors and $\Delta t \neq 0$, the point is not moving at
	  all considering these vectors are expressed with respect to an
	  inertial frame\footnote{For geostationary orbits, position vectors
	  remain the same as seen from a frame centered and rotating
          with the attractor. However, the orbiting body is seen to move
	  from a inertial frame.} centered in the attractor. This is not a valid
	  type of motion for the two-body dynamics.

  \item For a $180$ degrees transfer angle, an assumption needs to be done about
        the fundamental orbit of the plane, since $i$ nor $\Omega$ are defined.
        Some authors impose the $XY$ plane of the inertial reference frame to be
        the one containing all the motion of this particular problem.
\end{itemize}

Now that all possible transfer orbit scenarios have been introduced from the
point of view of geometry, it is possible to present the solution to the problem
from an analytical point of view.

