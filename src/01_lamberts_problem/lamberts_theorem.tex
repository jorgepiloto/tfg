\section{Lambert's theorem and implicit solution}
\label{sec:lamberts_theorem}

The Lambert's problem receives its name after Johann Heinrich Lambert formulated
it in a letter to Leonhard Euler, see APPENDIX XX for more information about
Lambert's problem origin and timeline.

Lambert proof in his \citetitle{lambert1761} that the time of flight between two
position vectors only depends on the sum of these radial distances, the linear
distance between them and the semi-major axis of the orbit. This is known as
\textit{the Lambert's theorem}.

\subsection{The vis-viva equation in its differential form}

To demonstrate Lambert's theorem, the original procedure devised by him is
presented in the following lines. Starting from the vis-viva equation in
\ref{eq:visviva}:

\begin{equation}
  v = \frac{dr}{dt} \sqrt{\frac{2\mu}{r} - \frac{\mu}{a}}
  \label{eq:visviva}
\end{equation}

it is possible to integrate to find that:

\begin{equation}
  \Delta t = \frac{1}{\sqrt{\mu}} \int_{s-c}^{s} \frac{r dr}{\sqrt{2r -
      r^{2}/a}}
  \label{eq:visviva_integral}
\end{equation}

where in previous equation $s$ is the semi-perimeter and $c$ the chord, being
given by:

\begin{equation}
  s = \frac{\norm{\vec{r_{1}}} + \norm{\vec{r_{2}}} +
    \norm{\vec{c}}}{2}\quad\quad\quad\quad
  \vec{c} = \vec{r_{2}} - \vec{r_{1}}
  \label{eq:s_and_c_lambert}
\end{equation}

\subsection{Series solution to the vis-viva differential equation}

Lambert provided equation \ref{eq:visviva_integral} in the form of a
series\footnote{This equation has been directly taken from \cite{battin1999}, in
  particular from problem 7-1 of the book.}: \ref{eq:visviva_series}:

\begin{flalign*}
  \Delta t = \frac{1}{\sqrt{\mu}} \left(\frac{\sqrt{2}}{3}[s^{\frac{3}{2}}\mp (s-c)^{\frac{3}{2}}] \right. &  &
\end{flalign*}

\vspace{-1.25cm}
\begin{equation}
  \left. +\sum_{n=1}^{\infty}\frac{\sqrt{2}}{2n+3}\frac{(2n-1)!}{2^{3n-1}n!(n-1)!}[s^{n+\frac{3}{2}} \mp (s-c)^{n+\frac{3}{2}}]\frac{1}{a^{n}}\right)
  \label{eq:visviva_series}
\end{equation}

However, notice that for solving the value of the semi-major axis a numerical
method is required, as equation \ref{eq:visviva_series} is not in explicit form
for $a$.

Lambert told Euler that he would devise the explicit form of
\ref{eq:visviva_series} (see \cite{albouy2019}) but Lambert's untimely death
prevented this.
