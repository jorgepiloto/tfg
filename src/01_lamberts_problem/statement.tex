\section{The statement of the problem}
\label{sec:statement}

Lambert's problem states to find for the Keplerian orbit which connects two
known position vectors, $\vec{r_{1}}$ and $\vec{r_{2}}$, over a finite amount of
time, $\Delta t$, under a gravity field of gravitational parameter $\mu$. The
three-dimensional representation of this problem was initially introduced by
figure \ref{fig:problem_geometry}. This problem is usually named \textit{the
  direct-arc transfer problem} and it was posed by Johann Heinrich Lambert in a
letter to Leonhard Euler as properly pointed by \cite{albouy2019}. Albouy made
an outstanding work by reviewing the whole problem's timeline from the very
beginning up to the modern days.

The direct transfer problem is the most common one when addressing the Lambert's
problem. However, it is also possible to specify the number of revolutions of
the orbiting body before reaching the final position vector. This type of
problem is known as \textit{the multi-revolution problem}.

It is important to say that perturbations will not be considered during this
whole work: only the Keplerian Lambert's problem is studied.

Because the answer to either the direct-arc transfer of the multi-revolution one
is an orbit, we then need a total of six parameters (3 translations + 3
rotations) to fully describe its shape, location and orientation. These set of
parameters is usually referred to as \textit{orbital elements}. Two of the most
common ones are:

\begin{itemize}
  \item RV set: composed by the three components of the initial position
        vector $\vec{r_{1}}$ and the other three components of the
        initial velocity vector $\vec{v_{1}}$.
  \item COE set: known as the \textit{classic orbit elements}. The
        elements which form this set are the semi-major axis $a$, the
        orbit's eccentricity $e$, its inclination $i$, right ascension
        of the ascending node $\Omega$, argument of the perigee $w$ and
        finally the true anomaly denoted by $\nu$.
\end{itemize}

Therefore, the Lambert's problem is solved once a full set of orbital elements
is retrieved, as the orbit between the two known position vectors is finally
known.
