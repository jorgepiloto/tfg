\section{Classification and inheritance diagram}

The amount of devised solutions along Lambert's problem timeline is massive.
However, after spotting some common points between different solvers, it is
possible to classify them according to the free-parameter employed, the
numerical method used, the initial guess procedure or the velocity vectors
construction. Among these four, the classification based on the free-parameter
is the most useful one, as it also allows to track the evolution of a particular
solver. 

In addition to the classification presented in the following lines, an
inheritance diagram has been produced so reader can have a better understanding
about the relations between the different published solvers over time. The work
by \cite{sangra2020} has been extremely useful and expanded.

\subsection{Semi-major axis based solvers}

The first methods to solve for the Lambert's problem made use of the semi-major
axis as the free-parameter, see previously introduced Lagrange solution in
section \ref{sec:lagrange_sol}. In addition, this variable is directly involved
in the Lambert's theorem (see section \ref{sec:lamberts_theorem}).

However, the analysis made by \cite{battin1999} demonstrates that iterating over
the semi-major axis is not convenient for the case of multiple revolutions,
where a pair of conjugate orbits exist. In addition to this, the derivative of
the transfer time becomes singular when the semi-major axis $a = a_m$, being
$a_m = s / 2$ the semi-major axis of the minimum energy orbit.

The first algorithm found in literature is the one by \cite{lagrange1788}. This
author set the basis for others to improve on his method. In particular,
\cite{prussing2000} expanded the solution to the multi-revolution case and
\cite{wailliez2014} improved the convergence of the method by making use of a
Householder's root solver over a simple-semi analytical expression for the time
of flight. Finally, \cite{jiang2016} revisits the problem for both the elliptic
and hyperbolic transfer types.

However, another branch within the semi-major axis solvers was started by
\cite{thorne1995} when this author devised a series solution for the Lambert's
problem. This algorithm was improved by the same author some years later (see
\cite{thorne2004}). A convergence analysis of the series was made again by him
in \cite{thorne2015}. Notice that, even if this algorithm does not require from
a numerical solver nor initial guess, it still makes use of a free-parameter and
the velocity construction method.

\vspace{0.5cm}
\begin{figure}[h]
  \centering
  \includegraphics[scale=1.00]{static/a_solvers.pdf}
  \caption{Inheritance diagram for semi-major axis based solvers. Two branches
  exist: the one devised by Lagrange and the series-based one by Thorne.}
  \label{fig:a_solvers}
\end{figure}


\subsection{Eccentricity based solvers}

Lambert's solvers which iterative over the eccentricity $e$ of the transfer are
relatively new if compared to other solvers. The first algorithm of this type in
history seems to be the one developed by \cite{escobal1965}. However,
\cite{Battin 1999} introduced in his book an important property of the
eccentricity vector within the context of Lambert's problem: its projection
along the chord-wise direction is kept constant. 

Previous statement can be proof from the orbit equation, where $\vec{e} \cdot
\vec{r} = p - r$. If evaluated at the two known position vectors, then $\vec{e}
\cdot (\vec{r_2} - \vec{r_1}) = \norm{\vec{r_1}} - \norm{\vec{r_2}}$. Because
$\vec{c} = \vec{r_2} - \vec{r_1}$, and the norm of the vectors does not change
over time, the projection $\vec{e} \cdot \vec{c} = e_{c}$ is seen to be
constant. Some authors also refer to $e_{c}$ as $e_F$.

The first author to devise a method based on previous property was
\cite{avanzini2008}, introducing a simple algorithm which iterates over the
transverse component of the eccentricity vector $e_T$, such that $e =
\sqrt{e_c^2 + e_T^2}$. However, this algorithm was improved by \cite{he2010} who
expanded to the multi-revolution case and provided the derivative of Kepler's
equation with respect to the free-parameter. Finally, new improvements were made
by \cite{wen2014} reducing the computational cost and therefore, increasing the
overall performance.

\vspace{0.5cm}
\begin{figure}[h]
  \centering
  \includegraphics[scale=1.00]{static/ecc_solvers.pdf}
  \caption{Inheritance diagram for eccentricity based solvers. Notice that the
  branch started by Avanzini is the most modern one and the mother of modern
  solvers based on this orbit parameter.}
  \label{fig:ecc_solvers}
\end{figure}

\subsection{True anomaly based solvers}

Regarding solvers which iterate over the true anomaly of the transfer orbit,
only a single one has been found among the whole Lambert's problem literature:
the one devised by \cite{gunkel1960}. This algorithm also appears in the book by
\citeauthor{escobal1965}.

\vspace{0.5cm}
\begin{figure}[h]
  \centering
  \includegraphics[scale=1.00]{static/nu_solvers.pdf}
  \caption{Inheritance diagram for true anomaly based solvers. Only one solver
  is found to belong to this set of solvers as iterating over $\nu$ is not the
  most popular way of addressing the problem.}
  \label{fig:nu_solvers}
\end{figure}


\subsection{Semi-latus rectum based solvers}

The semi-latus rectum $p$ of an orbit is usually called the orbital parameter and it
is related with the semi-major axis of the orbit and its eccentricity via $p =
a(1 - e^2)$. Therefore, the semi-latus rectum has been used by some authors as
the free-parameter.

We might establish \cite{gauss1809} the first one to use this parameter.
Although the original algorithm proposed by the genius does not iterate over
this variable, it is the first orbit element to be obtained after the numerical
process. Gauss relates the triangle to area sector to obtain a pair of equations
from which the Lambert's problem can be solved. In fact, he used the method to
the discovery of Ceres' orbit one year after it was spotted for the first time.
However, this method is only valid for angles lower than $90$ degrees
approximately and it is singular for transfer angles of $180$ degrees.

Some years later, \cite{battin1984} improved the convergence of Gauss' algorithm
by expanding it to all types of orbit in a universal variable approach.
\citeauthor{sangra2020} includes this algorithm in the universal branch of
solutions but we decided to included it here just to point out its inheritance
from the one devised by Gauss. The algorithm by Battin moves the singularity
from the $180$ to $360$, which is a more reasonable corner case. In addition, it
improves by far the accuracy of Gauss solver. Finally, Battin's algorithm was
improved to also solve for the multi-revolution scenario once the work of
\cite{shen2003} was published.

A new branch based on the $p$ parameter was started by \cite{bate1971}. Among
the set of algorithms published in Bate's book, the one iterating over the
semi-latus is simple to implement. However, this author did not imposed a
particular numerical method to solve for the transcendental equation. The only
method found in literature inheriting from Bate's one is the solver devised by
\cite{alhulayil2017} in which the authors make use of a 4th Taylor series
expansion. Although they do not cite Bate's solver, the variables used by them
are clearly the ones from Bate's book.


The last branch was started by \cite{herrick1959}. This differs from Bate's
solver as it involves the eccentricity during the computation of the
free-parameter.


\vspace{0.5cm}
\begin{figure}[h]
  \centering
  \includegraphics[scale=1.00]{static/p_solvers.pdf}
  \caption{Inheritance diagram for semi-latus rectum based solvers. A total of
  three branches have been devised along history for addressing the problem
  making use of this free-parameter.}
  \label{fig:p_Solvers}
\end{figure}


\subsection{Universal formulation based solvers}

The usage universal formulation as the free-parameter holds the greatest amount
of solvers and the deepest inheritance diagram. In fact, this approach to solve
the problem is considered to be the origin of modern Lambert's solvers with the
publication of \cite{lancaster1970} in which an elegant figure relating the
free-parameter and the non-dimensional time of flight appeared for the first
time. This algorithm was able to cover all types of transfer orbits (elliptic,
parabolic and hyperbolic) and also capable of solving the multi-revolution
problem.

Lancaster set the basis for other authors to improve the method and
\cite{gooding1990} pick up the baton by introducing one of the most popular
algorithms. In his work, Gooding provided Fortran77 routines so other authors
could easily reproduce his work. However, he imposed a total of three iterations
after checking that no more were required to obtain great accuracy in the
results.

Even when this branch was not expected to be improved, \cite{izzo2015} proposed
his famous solver by introducing one last change of variable and upgrading the
numerical method employed to a Householder's one, as opposite to Gooding who
employed Halley's method.

The second branch within solvers using universal formulation was started by
\cite{bate1971}. Again, in his book, this author introduced a new solver.
No root solver was imposed although bisection and Newton's method were cited.
However, some years later \cite{vallado2013} improved the convergence of the
method by imposing a bisection method which, as opposite to Newton's method,
converges always to the expected value.

Finally, a great progress was made by \cite{arora2013} by introducing an smart
cosine transformation. This new algorithm was capable of solving the
multi-revolution problem and employed an strong rational formulae initial guess.

\vspace{0.5cm}
\begin{figure}[h]
  \centering
  \includegraphics[scale=1.00]{static/universal_solvers.pdf}
  \caption{Inheritance diagram for universal formulation solvers based solvers.
  The two branches initiated by Bate and Lancaster are visible.} 
  \label{fig:universal_solvers}
\end{figure}


\subsection{Regularizing transformation based solvers}

Solvers based on regularized transformations are not abundant in literature. The
main goal of these solvers is to avoid the singularities in the Lambert's
problem by moving it from $\Re^{N} \rightarrow \Re^{N+1}$. At least three types of
regularizing transformations are known:

\begin{itemize}
  \item Levi-Civita.
  \item Kustaanheimo-Stiefel.
  \item Radial-Inversion.
\end{itemize}

The mathematical background for the regularizing transformation is out of the
scope of this work but reader is encouraged to refer to Chapter 3 of
\cite{celletti2002} book's.

The first known solver to take advantage of one of the transformations is
\cite{simo1973}. The algorithm by this author makes use of the Levi-Civita,
covering degenerate conics too. This algorithm was enhanced by \cite{torre2018}
by introducing the multi-revolution scenario and enhancing the initial guess routine.

Another branch found in literature using the regularizing transformation was
started by \cite{kriz1976}. As opposite to previous solvers, this algorithm
makes use of the Kustaanheimo-Stiefel transformation. That same year,
\cite{jezewski1976} published another working on the same principle as the one
suggested by Kriz.

\vspace{0.5cm}
\begin{figure}[h]
  \centering
  \includegraphics[scale=1.00]{static/regularizing_solvers.pdf}
  \caption{Inheritance diagram for regularized transformation solvers. Only the
  transformations via Levi-Civita and Kustaanheimo-Stiefel are the ones serving
  as base for these type of solvers.} 
  \label{fig:regularizing_solvers}
\end{figure}


\subsection{Flight path angle based}

The flight path angle $\gamma = \arccos{\left(\frac{r_p v_p}{r v} \right)}$,
where the sub-index $p$ denotes the perpendicular direction to the current radial
distance, is the last of the variables used as free-parameter. We already
discussed about the applications of Lambert's problem within the rendezvous and
intercepting maneuvers. Algorithms using the flight path angle were usually
associated with ballistic missiles, benefiting from previous commented
applications.

\cite{wheelon1959} was the first to publish a solver using the flight path
angle. He published his article not referring to the Lambert's problem but to
the determination of the trajectory of ballistic missiles. Therefore, this is a
case were the IOD problem is solved via the BVP.

Nevertheless, some years later \cite{nelson1992} published his article about the
solution of Lambert's problem explicitly. By iterating over the flyout angle,
Nelson's solver computes the initial velocity vector so the full orbit is
determined.

Another contribution to this set of solvers was made by \cite{arlulkar2011}.
This algorithm inherits from both previous ones. He also provided the
multi-revolution scenario. 

Finally, the last algorithm of this set was devised by \cite{ahn2013}. Although
using the flight path angle as free-parameter, the main feature of this solver
is the usage of analytic gradients as the way of computing the solution.

\vspace{0.5cm}
\begin{figure}[h]
  \centering
  \includegraphics[scale=0.9]{static/gamma_solvers.pdf}
  \caption{Inheritance diagram for flight path angle based solvers. Only a
  single branch is known within this set of solvers.} 
  \label{fig:regularizing_solvers}
\end{figure}


