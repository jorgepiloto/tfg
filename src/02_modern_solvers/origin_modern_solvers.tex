\section{Origin of modern solvers}

The applications of Lambert's problem were presented in section
\ref{sec:applications}, where it was seen that targeting and maneuvering were
the most interesting ones apart from IOD. However it was not till the 60s, with
the space-race, when the problem was applied not only from a mathematical point
of view but from a real one.

Missions requiring docking or rendezvous required to solve for the Lambert's
problem as a first approach to the phasing problem\footnote{Phasing problem can
be explained as the problem which combines achieving a particular orbit but also
a given true anomaly on it.}.

The space-race era shared its time with the development of the first electronic
computers\footnote{Analog/mechanical, digital and quantum computers are the main
three families.} In fact, the first aerospace project/program to make use of
integrated circuit boards (ICB) was the Apollo one, in particular in its
guidance computer. In order to ease the usage of computers by scientists and
engineers, a high-level programming language named Fortran\footnote{Fortran
means \textit{Formula Translating System} and became really popular
between the 80s and 90s. Lots of legacy code are written in this
language.} was created in 1975.

All previous facts, in the context of developing new procedures for the
Lambert's problem, lead to what are known as \textit{modern Lambert's solvers}.
These new solvers were developed in the form of computer algorithms, taking
advantage of different numerical methods for finding in a quick way the
solution to the problem and avoiding its singularities. In addition, computers
allowed to plot contour plots in a more easy and interactive way, giving a deep
understanding not only on Lambert's problem but in many other ones.

The computational power grow originally at the rate predicted by Moore in his
famous law\footnote{Moore's law states that the amount of transistors per
surface unit doubles every two years.}, although nowadays the Huang's
law applies\footnote{Huang's law claims that GPU development is growing
faster than that of the central processing units (CPUs), making Moore's
law obsolete.} The effect of these laws can be seen mainly in the lower amount
of time a today's computer requires to execute a piece of code as opposite to
the initial days of space-race and computers era. Nevertheless, there are still
efforts to increase computers performance so more complex problems in many
different scientific fields can be explored, understood and solved.

Therefore, it is not strange that the amount of published modern solutions, as
early introduced by figure \ref{fig:art_lambert}, has increased during the last
decades and most of the new techniques and procedures try to beat the ones
developed during the 60s.

